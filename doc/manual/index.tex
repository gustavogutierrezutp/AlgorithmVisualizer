% Options for packages loaded elsewhere
% Options for packages loaded elsewhere
\PassOptionsToPackage{unicode}{hyperref}
\PassOptionsToPackage{hyphens}{url}
\PassOptionsToPackage{dvipsnames,svgnames,x11names}{xcolor}
%
\documentclass[
  spanish,
  letterpaper,
  DIV=11,
  numbers=noendperiod]{scrreprt}
\usepackage{xcolor}
\usepackage{amsmath,amssymb}
\setcounter{secnumdepth}{5}
\usepackage{iftex}
\ifPDFTeX
  \usepackage[T1]{fontenc}
  \usepackage[utf8]{inputenc}
  \usepackage{textcomp} % provide euro and other symbols
\else % if luatex or xetex
  \usepackage{unicode-math} % this also loads fontspec
  \defaultfontfeatures{Scale=MatchLowercase}
  \defaultfontfeatures[\rmfamily]{Ligatures=TeX,Scale=1}
\fi
\usepackage{lmodern}
\ifPDFTeX\else
  % xetex/luatex font selection
\fi
% Use upquote if available, for straight quotes in verbatim environments
\IfFileExists{upquote.sty}{\usepackage{upquote}}{}
\IfFileExists{microtype.sty}{% use microtype if available
  \usepackage[]{microtype}
  \UseMicrotypeSet[protrusion]{basicmath} % disable protrusion for tt fonts
}{}
\makeatletter
\@ifundefined{KOMAClassName}{% if non-KOMA class
  \IfFileExists{parskip.sty}{%
    \usepackage{parskip}
  }{% else
    \setlength{\parindent}{0pt}
    \setlength{\parskip}{6pt plus 2pt minus 1pt}}
}{% if KOMA class
  \KOMAoptions{parskip=half}}
\makeatother
% Make \paragraph and \subparagraph free-standing
\makeatletter
\ifx\paragraph\undefined\else
  \let\oldparagraph\paragraph
  \renewcommand{\paragraph}{
    \@ifstar
      \xxxParagraphStar
      \xxxParagraphNoStar
  }
  \newcommand{\xxxParagraphStar}[1]{\oldparagraph*{#1}\mbox{}}
  \newcommand{\xxxParagraphNoStar}[1]{\oldparagraph{#1}\mbox{}}
\fi
\ifx\subparagraph\undefined\else
  \let\oldsubparagraph\subparagraph
  \renewcommand{\subparagraph}{
    \@ifstar
      \xxxSubParagraphStar
      \xxxSubParagraphNoStar
  }
  \newcommand{\xxxSubParagraphStar}[1]{\oldsubparagraph*{#1}\mbox{}}
  \newcommand{\xxxSubParagraphNoStar}[1]{\oldsubparagraph{#1}\mbox{}}
\fi
\makeatother


\usepackage{longtable,booktabs,array}
\newcounter{none} % for unnumbered tables
\usepackage{calc} % for calculating minipage widths
% Correct order of tables after \paragraph or \subparagraph
\usepackage{etoolbox}
\makeatletter
\patchcmd\longtable{\par}{\if@noskipsec\mbox{}\fi\par}{}{}
\makeatother
% Allow footnotes in longtable head/foot
\IfFileExists{footnotehyper.sty}{\usepackage{footnotehyper}}{\usepackage{footnote}}
\makesavenoteenv{longtable}
\usepackage{graphicx}
\makeatletter
\newsavebox\pandoc@box
\newcommand*\pandocbounded[1]{% scales image to fit in text height/width
  \sbox\pandoc@box{#1}%
  \Gscale@div\@tempa{\textheight}{\dimexpr\ht\pandoc@box+\dp\pandoc@box\relax}%
  \Gscale@div\@tempb{\linewidth}{\wd\pandoc@box}%
  \ifdim\@tempb\p@<\@tempa\p@\let\@tempa\@tempb\fi% select the smaller of both
  \ifdim\@tempa\p@<\p@\scalebox{\@tempa}{\usebox\pandoc@box}%
  \else\usebox{\pandoc@box}%
  \fi%
}
% Set default figure placement to htbp
\def\fps@figure{htbp}
\makeatother



\ifLuaTeX
\usepackage[bidi=basic,shorthands=off]{babel}
\else
\usepackage[bidi=default,shorthands=off]{babel}
\fi
\ifLuaTeX
  \usepackage{selnolig} % disable illegal ligatures
\fi


\setlength{\emergencystretch}{3em} % prevent overfull lines

\providecommand{\tightlist}{%
  \setlength{\itemsep}{0pt}\setlength{\parskip}{0pt}}



 


\KOMAoption{captions}{tableheading}
\makeatletter
\@ifpackageloaded{tcolorbox}{}{\usepackage[skins,breakable]{tcolorbox}}
\@ifpackageloaded{fontawesome5}{}{\usepackage{fontawesome5}}
\definecolor{quarto-callout-color}{HTML}{909090}
\definecolor{quarto-callout-note-color}{HTML}{0758E5}
\definecolor{quarto-callout-important-color}{HTML}{CC1914}
\definecolor{quarto-callout-warning-color}{HTML}{EB9113}
\definecolor{quarto-callout-tip-color}{HTML}{00A047}
\definecolor{quarto-callout-caution-color}{HTML}{FC5300}
\definecolor{quarto-callout-color-frame}{HTML}{acacac}
\definecolor{quarto-callout-note-color-frame}{HTML}{4582ec}
\definecolor{quarto-callout-important-color-frame}{HTML}{d9534f}
\definecolor{quarto-callout-warning-color-frame}{HTML}{f0ad4e}
\definecolor{quarto-callout-tip-color-frame}{HTML}{02b875}
\definecolor{quarto-callout-caution-color-frame}{HTML}{fd7e14}
\makeatother
\makeatletter
\@ifpackageloaded{bookmark}{}{\usepackage{bookmark}}
\makeatother
\makeatletter
\@ifpackageloaded{caption}{}{\usepackage{caption}}
\AtBeginDocument{%
\ifdefined\contentsname
  \renewcommand*\contentsname{Tabla de contenidos}
\else
  \newcommand\contentsname{Tabla de contenidos}
\fi
\ifdefined\listfigurename
  \renewcommand*\listfigurename{Listado de Figuras}
\else
  \newcommand\listfigurename{Listado de Figuras}
\fi
\ifdefined\listtablename
  \renewcommand*\listtablename{Listado de Tablas}
\else
  \newcommand\listtablename{Listado de Tablas}
\fi
\ifdefined\figurename
  \renewcommand*\figurename{Figura}
\else
  \newcommand\figurename{Figura}
\fi
\ifdefined\tablename
  \renewcommand*\tablename{Tabla}
\else
  \newcommand\tablename{Tabla}
\fi
}
\@ifpackageloaded{float}{}{\usepackage{float}}
\floatstyle{ruled}
\@ifundefined{c@chapter}{\newfloat{codelisting}{h}{lop}}{\newfloat{codelisting}{h}{lop}[chapter]}
\floatname{codelisting}{Listado}
\newcommand*\listoflistings{\listof{codelisting}{Listado de Listados}}
\makeatother
\makeatletter
\makeatother
\makeatletter
\@ifpackageloaded{caption}{}{\usepackage{caption}}
\@ifpackageloaded{subcaption}{}{\usepackage{subcaption}}
\makeatother
\newcounter{quartocallouttipno}
\newcommand{\quartocallouttip}[1]{\refstepcounter{quartocallouttipno}\label{#1}}
\usepackage{bookmark}
\IfFileExists{xurl.sty}{\usepackage{xurl}}{} % add URL line breaks if available
\urlstyle{same}
\hypersetup{
  pdftitle={DSViz: Listas enlazadas},
  pdfauthor={Gustavo Gutiérrez-Sabogal; Jovanny Bedoya-Guapacha; Nancy Janet Castillo-Rodríguez},
  pdflang={es},
  colorlinks=true,
  linkcolor={blue},
  filecolor={Maroon},
  citecolor={Blue},
  urlcolor={Blue},
  pdfcreator={LaTeX via pandoc}}


\title{DSViz: Listas enlazadas}
\usepackage{etoolbox}
\makeatletter
\providecommand{\subtitle}[1]{% add subtitle to \maketitle
  \apptocmd{\@title}{\par {\large #1 \par}}{}{}
}
\makeatother
\subtitle{Versión 1.0: Visualización e interacción con listas enlazadas}
\author{Gustavo Gutiérrez-Sabogal \and Jovanny
Bedoya-Guapacha \and Nancy Janet Castillo-Rodríguez}
\date{2026-02-16}
\begin{document}
\maketitle

\renewcommand*\contentsname{Tabla de contenidos}
{
\setcounter{tocdepth}{2}
\tableofcontents
}
\listoffigures

\bookmarksetup{startatroot}

\chapter{Introducción}\label{introducciuxf3n}

Las \textbf{estructuras de datos} son uno de los pilares del desarrollo
de software. Estas definen como la información (los datos) son
almacenados de manera eficiente para ser utilizados por los
\textbf{algoritmos}. Es por esta razón que su enseñanza y aprendizaje
constituyen un reto dentro de la carrera de ingeniería de sistemas.

La idea de crear un simulador tiene diferentes fundamentos:

\begin{itemize}
\item
  Proveer a los profesores del curso con una herramienta que permita
  mostrar ideas y conceptos fundamentales a los estudiantes de una
  manera clara, precisa y eficaz.
\item
  Permitir a los estudiantes tener un recurso de aprendizaje que puedan
  utilizar durante la clase y también de manera asíncrona para reforzar,
  entender y verificar los conceptos impartidos durante el curso.
\item
  Recopilar lo que por muchas iteraciones del curso de estructuras de
  datos se ha constituido como material de estudio. Esto no solo
  comprende código fuente sino también valoración sobre los aspectos
  particulares de cada estructura de datos en donde se ha evidenciado
  que nuestros estudiantes presentan mayores inconvenientes.
\end{itemize}

\section{Instalación}\label{instalaciuxf3n}

El software aquí descrito se puede acceder a través del siguiente
enlace:

\url{https://gustavogutierrezutp.github.io/AlgorithmVisualizer/sll}

Después de hacer click en el enlace, se abrirá el navegador en una
página como la que se presenta a continuación.

\pandocbounded{\includegraphics[keepaspectratio]{../figures/random5.png}}

Como se trata de una herramienta pedagógica que tiene como objetivo ser
fácilmente accesible los autores optaron por la manera más sencilla y
con cero requisitos de instalación. Todo corre completamente en el
navegador y no requiere de ninguna otra dependencia. En el
Capítulo~\ref{sec-funcionalidad} se describe en detalle el
funcionamiento de la aplicación.

\bookmarksetup{startatroot}

\chapter{Funcionalidad}\label{sec-funcionalidad}

Antes de comenzar a describir cada una de las funcionalidades de la
aplicación es importante describir la interfaz de usuario y las partes
que la componen. Cuando el usuario (estudiante o profesor) ejecuta la
aplicación por primera vez se encuentra con una interfaz como la
presentada en la Figura~\ref{fig-main-window}.

\begin{figure}

\centering{

\pandocbounded{\includegraphics[keepaspectratio]{../figures/biglist.png}}

}

\caption{\label{fig-main-window}Ventana principal con lista de ejemplo
desplegada.}

\end{figure}%

\section{Interfaz de usuario}\label{interfaz-de-usuario}

La Figura~\ref{fig-main-window-annotated} presenta cada una de las
partes en las que está dividida la interfaz de usuario:

\begin{enumerate}
\def\labelenumi{\arabic{enumi}.}
\tightlist
\item
  Barra de navegación. Aquí se encuentran enlaces a la documentación y
  al repositorio con el código fuente. También un enlace a un tour que
  lleva al usuario por cada uno de los componentes y funcionalidades más
  importantes de la aplicación.
\item
  Lienzo. Esta es la parte donde toda la visualización tiene lugar. Es
  el lugar donde se visualizará la lista y se presentaran las
  animaciones de las operaciones según sean seleccionadas por el
  usuario.
\item
  Menú de operaciones. En este lugar se encuentran de manera agrupada
  por funcionalidad las acciones que el usuario puede realizar en la
  aplicación.
\end{enumerate}

El flujo de trabajo que se espera por parte del usuario está enfocado a
utilizar las partes 2 y 3 de la interfaz de usuario. La idea es que este
último selecciona la operación en la parte 3 y visualiza su ejecución en
la parte 2.

\begin{figure}

\centering{

\pandocbounded{\includegraphics[keepaspectratio]{../figures/main-annotated.png}}

}

\caption{\label{fig-main-window-annotated}Partes principales de la
interfaz de usuario.}

\end{figure}%

Las operaciones soportadas por la aplicación están divididas por
funcionalidad y presentadas de esta manera en la parte 3:

\begin{itemize}
\tightlist
\item
  Creación. Comprenden las diferentes formas en que se puede crear una
  lista enlazada. Entre estas formas se incluye: vacía, de una secuencia
  de números o de manera aleatoria.
\item
  Operaciones sobre la estructura de datos. Son las operaciones que le
  permiten al programador interactuar con la estructura de datos. En
  esta sección se incluyen operaciones de inserción, acceso y
  eliminación entre otras.
\item
  Herramientas de programación. Permiten la visualización de componentes
  de programación como punteros. Estos elementos no son conceptuales
  pero si son relevantes a la hora de la implementación de la estructura
  de datos.
\item
  Opciones de visualización. Permiten al usuario alterar algunas
  características de la visualización de la estructura de datos. Estas
  opciones son útiles para la presentación y visualización de la
  estructura de datos en entornos de estudio o de docencia.
\end{itemize}

A continuación describiremos cada una de las partes. La parte superior
contiene una barra con el título y algunos enlaces de interés. Tal vez
el más importante es el botón con el texto \textbf{Tour} en la parte
superior derecha. Pulsar este botón activará una especie de tour por
cada uno de los elementos de la interfaz gráfica y le proporcionará al
usuario una idea de la funcionalidad de cada parte.

Las otras dos partes son las más importantes para la interacción con el
programador. En la sección vertical izquierda se puede apreciar una
barra de menús. En el caso de la figura todos ellos se encuentran
colapsados y entre otros, presentan los títulos: \textbf{List Creation},
\textbf{Operations}, \textbf{Programmer Tools} y \textbf{Display
Options}. Finalmente en la sección vertical derecha se encuentra el
lienzo que es el lugar donde aparecen las representaciones gráficas. En
este caso se aprecia una lista de números. En las secciones siguientes
se muestra detalladamente cada una de las funcionalidades.

\section{Visualización de la lista}\label{visualizaciuxf3n-de-la-lista}

Después de creada una lista, está será visualizada en el lienzo que hay
en la parte derecha. Por ejemplo, en la Figura~\ref{fig-rand5-commented}
se muestra la interfaz con una lista creada de 7 elementos. En esta
figura se encuentran las anotaciones de los diferentes nodos de la lista
y los componentes de uno de ellos.

\begin{figure}

\centering{

\pandocbounded{\includegraphics[keepaspectratio]{../figures/random5-commented.png}}

}

\caption{\label{fig-rand5-commented}Imagen anotada de la visualización
de una lista enlazada de 5 elementos en el lienzo.}

\end{figure}%

Existen diferentes formas de interactuar con la representación visual de
la lista enlazada. Los nodos de la lista pueden ser reposicionados y su
encadenamiento puede ser alterado. En la parte inferior izquierda del
lienzo se encuentra una pequeña barra vertical con botones que tienen
acción sobre la visualización del lienzo.

\begin{tcolorbox}[enhanced jigsaw, colframe=quarto-callout-tip-color-frame, colback=white, leftrule=.75mm, colbacktitle=quarto-callout-tip-color!10!white, breakable, toprule=.15mm, opacityback=0, opacitybacktitle=0.6, coltitle=black, bottomtitle=1mm, bottomrule=.15mm, arc=.35mm, toptitle=1mm, titlerule=0mm, title=\textcolor{quarto-callout-tip-color}{\faLightbulb}\hspace{0.5em}{Tip \ref*{tip-canvas-toolbar}: Consejos}, left=2mm, rightrule=.15mm]

\quartocallouttip{tip-canvas-toolbar} 

\begin{figure}[H]

\begin{minipage}{0.10\linewidth}
\pandocbounded{\includegraphics[keepaspectratio]{../figures/xyflow-toolbar.png}}\end{minipage}%
%
\begin{minipage}{0.90\linewidth}

\begin{itemize}
\tightlist
\item
  El tercer botón de arriba a abajo permite siempre enfocar todos los
  elementos dentro del lienzo. Esto es de gran importancia porque
  facilita mucho la interacción con la visualización.
\item
  El cuarto botón con el símbolo
  \pandocbounded{\includegraphics[keepaspectratio]{../figures/lock.png}}
  protege el lienzo de cualquier tipo de interacción. Esta opción es muy
  útil cuando se desean tomar imágenes de captura de pantalla o realizar
  explicaciones sin que se altere la visualización.
\end{itemize}

\end{minipage}%

\end{figure}%

\end{tcolorbox}

Cada nodo de la lista consta de dos partes como se identifican en la
Figura~\ref{fig-rand5-commented}. La primera contiene el dato y la
segunda representa el apuntador al siguiente nodo en la lista. Cuando el
\emph{mouse} se posiciona sobre este último el apuntador es resaltado en
un color diferente (ver Figura~\ref{fig-next-pointer}). Esto es útil
cuando las listas no se muestran de manera lineal. A continuación se
muestra la misma lista después de ser reorganizada arrastrando los nodos
con el mouse sobre el lienzo. El puntero se posicionó sobre el nodo que
contiene el valor \(44\).

\begin{figure}

\centering{

\pandocbounded{\includegraphics[keepaspectratio]{../figures/next-pointer.png}}

}

\caption{\label{fig-next-pointer}Visualización del puntero al siguiente
elemento de un nodo. Esta visualización se logra al colocar el puntero
del ratón sobre el apuntador del nodo.}

\end{figure}%

\bookmarksetup{startatroot}

\chapter{Operaciones}\label{sec-operaciones}

En este capítulo se detallan cada una de las operaciones disponibles a
través de la interfaz de usuario.

\section{Creación}\label{creaciuxf3n}

En el simulador las listas pueden ser creadas de diferentes maneras.
Todas las opciones de creación se encuentran en el primer sub menú con
el título \textbf{List Creation}.

\subsection*{Lista vacía}\label{lista-vacuxeda}
\addcontentsline{toc}{subsection}{Lista vacía}

\begin{figure}

\centering{

\pandocbounded{\includegraphics[keepaspectratio]{../figures/creation-empty.png}}

}

\caption{\label{fig-creation-empty}Interfaz para la creación de una
lista vacía.}

\end{figure}%

Para crear una lista vacía donde el usuario pueda comenzar a añadir
elementos haciendo uso de las operaciones se debe hacer click en la
pestaña \emph{Empty} y a continuación hacer click en el botón
\emph{Create Empty List} (ver Figura~\ref{fig-creation-empty}). Al dar
click en él aparecerá un mensaje en el lienzo de la derecha informando
que la lista actualmente visualizada se encuentra vacía.

\subsection*{Lista con números
aleatorios}\label{lista-con-nuxfameros-aleatorios}
\addcontentsline{toc}{subsection}{Lista con números aleatorios}

La siguiente opción para crear una lista es con elementos aleatorios.
Esto es especialmente útil cuando el usuario desea ver como se comporta
una operación sobre una lista que ya contiene elementos y no desea
insertarlos uno a uno. En esta opción es cuestión de seleccionar el
número de nodos utilizando el elemento deslizante y posteriormente
pulsar el botón \emph{Generate Random} (ver
Figura~\ref{fig-creation-random}). Al presionar el botón aparecerá la
lista en el lienzo de la derecha junto con sus elementos.

\begin{figure}

\centering{

\pandocbounded{\includegraphics[keepaspectratio]{../figures/creation-random.png}}

}

\caption{\label{fig-creation-random}Interfaz para la creación de una
lista inicializada de forma aleatoria.}

\end{figure}%

\subsection*{Lista personalizada}\label{lista-personalizada}
\addcontentsline{toc}{subsection}{Lista personalizada}

Cuando se requiere una lista con elementos específicos la tercera opción
es la indicada. Haciendo click en \emph{Custom} aparecerá un cuadro de
texto para insertar cada uno de los elementos de la lista (ver
Figura~\ref{fig-creation-custom}). El formato es simple: los elementos
deben estar encerrados por los símbolos \texttt{{[}}y \texttt{{]}}.
Adicionalmente los números deben estar separado por comas. Por ejemplo,
al introducir \texttt{{[}1,2,3,4,5{]}} se creará y visualizará la lista
que contiene los elementos del \(1\) al \(5\).

\begin{figure}

\centering{

\pandocbounded{\includegraphics[keepaspectratio]{../figures/creation-custom.png}}

}

\caption{\label{fig-creation-custom}Interfaz para la creación de una
lista con elementos puntuales.}

\end{figure}%

\section{Operaciones sobre la lista}\label{operaciones-sobre-la-lista}

Después de tener la lista creada o una lista vacía podemos comenzar a
realizar operaciones sobre ella. Esto con el fin de observar el
comportamiento de cada una de ellas. Las operaciones disponibles se
pueden acceder desde el menú \textbf{Operations} (ver
Figura~\ref{fig-operations}).

\begin{figure}

\centering{

\pandocbounded{\includegraphics[keepaspectratio]{../figures/operations-menu.png}}

}

\caption{\label{fig-operations}Panel de operaciones}

\end{figure}%

Las operaciones se encuentran agrupadas según su lógica. Por ejemplo, el
menú \textbf{Insert} agrupa las 4 operaciones que adicionan elementos en
la lista. El menú \textbf{Access} contiene 3 operaciones que acceden a
los elementos de la lista y así sucesivamente.

A continuación se describen en detalle cada una de las operaciones y su
visualización.

\subsection{Inserción}\label{inserciuxf3n}

La aplicación provee visualización para 4 diferentes tipos de inserción
a través del menú \textbf{insert} mostrado en la figura
Figura~\ref{fig-operations-only-insert}. Para todos los casos, el valor
a insertar debe ser definido en el primer cuadro de texto. En la figura
este presenta el valor \(25\). También dentro del mismo cuadro de texto
hay un botón que opcionalmente genera un valor aleatorio (ver
Tip~\ref{tip-rnd}).

\begin{figure}

\centering{

\pandocbounded{\includegraphics[keepaspectratio]{../figures/only-insert.png}}

}

\caption{\label{fig-operations-only-insert}Operaciones de inserción.}

\end{figure}%

\begin{tcolorbox}[enhanced jigsaw, colframe=quarto-callout-tip-color-frame, colback=white, leftrule=.75mm, colbacktitle=quarto-callout-tip-color!10!white, breakable, toprule=.15mm, opacityback=0, opacitybacktitle=0.6, coltitle=black, bottomtitle=1mm, bottomrule=.15mm, arc=.35mm, toptitle=1mm, titlerule=0mm, title=\textcolor{quarto-callout-tip-color}{\faLightbulb}\hspace{0.5em}{Tip \ref*{tip-rnd}: Consejo}, left=2mm, rightrule=.15mm]

\quartocallouttip{tip-rnd} 

\begin{figure}[H]

\begin{minipage}{0.10\linewidth}
\pandocbounded{\includegraphics[keepaspectratio]{../figures/rnd-button.png}}\end{minipage}%
%
\begin{minipage}{0.90\linewidth}
Pulsando este botón que se encuentra al lado del cuadro de texto se
generará un valor aleatorio. Este valor será insertado por cualquiera de
las operaciones.\end{minipage}%

\end{figure}%

\end{tcolorbox}

\subsubsection*{\texorpdfstring{\textbf{At
Head}}{At Head}}\label{at-head}
\addcontentsline{toc}{subsubsection}{\textbf{At Head}}

Pulsando el botón \textbf{At Head} se inserta el nuevo elemento al
inicio de la lista. El nuevo nodo se posicionará al inicio de la lista y
tendrá un color diferente (ver Tip~\ref{tip-cambiar-color}). La
Figura~\ref{fig-insert_at_head_before} muestra el lienzo antes de la
inserción. Finalmente la Figura~\ref{fig-insert_at_head_after} muestra
el lienzo después de la inserción.

\begin{figure}

\centering{

\pandocbounded{\includegraphics[keepaspectratio]{../figures/insert-head-before.png}}

}

\caption{\label{fig-insert_at_head_before}Vista anterior a la inserción
del valor \(25\) al inicio de la lista.}

\end{figure}%

\begin{figure}

\centering{

\pandocbounded{\includegraphics[keepaspectratio]{../figures/insert-head-after.png}}

}

\caption{\label{fig-insert_at_head_after}Vista posterior a la inserción
del valor \(25\) al inicio de la lista.}

\end{figure}%

\begin{tcolorbox}[enhanced jigsaw, colframe=quarto-callout-tip-color-frame, colback=white, leftrule=.75mm, colbacktitle=quarto-callout-tip-color!10!white, breakable, toprule=.15mm, opacityback=0, opacitybacktitle=0.6, coltitle=black, bottomtitle=1mm, bottomrule=.15mm, arc=.35mm, toptitle=1mm, titlerule=0mm, title=\textcolor{quarto-callout-tip-color}{\faLightbulb}\hspace{0.5em}{Tip \ref*{tip-cambiar-color}: Consejo}, left=2mm, rightrule=.15mm]

\quartocallouttip{tip-cambiar-color} 

\begin{figure}[H]

\begin{minipage}{0.30\linewidth}
\pandocbounded{\includegraphics[keepaspectratio]{../figures/apply-color.png}}\end{minipage}%
%
\begin{minipage}{0.70\linewidth}
En cualquier momento es posible cambiar el color de los nodos de la
lista utilizando la funcionalidad del menú \textbf{Display Options}
presentado a la izquierda. En este caso se seleccionan en el lienzo los
nodos a los que se les desea cambiar el color. Posteriormente se
selecciona el color que aparece al lado del botón \textbf{Apply} y se
pulsa este último.\end{minipage}%

\end{figure}%

\end{tcolorbox}

\subsubsection*{\texorpdfstring{\textbf{At
Position}}{At Position}}\label{at-position}
\addcontentsline{toc}{subsubsection}{\textbf{At Position}}

Para insertar un elemento en una posición específica se debe seleccionar
la posición deseada en el cuadro de texto al lado del botón \textbf{At
Position}. Posteriormente se debe pulsar el botón mencionado para
insertar el elemento en la posición seleccionada. En caso de ingresar
una posición mayor a la cantidad de elementos de la lista, la entrada de
texto se pondrá de color rojo indicando el error (ver
Figura~\ref{fig-insert-at-position-error}).

\begin{figure}

\centering{

\pandocbounded{\includegraphics[keepaspectratio]{../figures/insert-error.png}}

}

\caption{\label{fig-insert-at-position-error}Error al elegir una
posición no válida}

\end{figure}%

\subsubsection*{\texorpdfstring{\textbf{At
Tail}}{At Tail}}\label{at-tail}
\addcontentsline{toc}{subsubsection}{\textbf{At Tail}}

Para insertar una elemento al final de la lista existen dos formas. En
ambos casos los botones tienen el mismo nombre: \_\_At Tail\_. Uno de
ellos tiene un símbolo de advertencia en la parte superior derecha y el
otro no. Ambos insertan el elemento al final de la lista. La diferencia
es que uno lo hace recorriendo la lista desde el primer elemento y el
otro lo hace utilizando la referencia al último elemento de la lista.
Cuando se usa la primera opción se verá la animación en el lienzo de
como la lista es recorrida elemento a elemento.

\subsection{Borrado}\label{borrado}

Para el borrado de elementos existen tres formas diferentes y se
muestran en Figura~\ref{fig-operations-only-remove}. El botón con título
\textbf{Head} permite borrar el primer elemento de la lista. De manera
similar el botón con título \textbf{Tail} permite borrar el último
elemento de la lista. Por último el botón con título \textbf{Position}
permite borrar un elemento en una posición específica. En este caso
primero se debe seleccionar la posición deseada en el cuadro de texto al
lado del mismo.

\begin{figure}

\centering{

\pandocbounded{\includegraphics[keepaspectratio]{../figures/only-remove.png}}

}

\caption{\label{fig-operations-only-remove}Operaciones de borrado.}

\end{figure}%

Cuando se borra utilizando una posición no válida, el cuadro de texto se
pondrá de color rojo indicando el error de manera similar al caso de la
inserción.

\subsection{Acceso}\label{acceso}

El siguiente grupo de operaciones es el de acceso. Las operaciones de
acceso son tres:

\begin{itemize}
\tightlist
\item
  \textbf{Front}: el primer elemento de la lista.
\item
  \textbf{Back}: el último elemento de la lista.
\item
  \textbf{Nth}: el elemento en una posición específica.
\end{itemize}

Para cada una de las operaciones el elemento se identificará de manera
gráfica en el lienzo.

La Figura~\ref{fig-operations-only-access} muestra las operaciones de
acceso descritas anteriormente. En el caso de \textbf{Nth} es necesario
insertar la posición en el cuadro de texto. Cuando se da click en el
botón, se visualizará el recorrido hasta llegar a la posición deseada.
Si se inserta una posición no válida, el cuadro de texto se pondrá de
color rojo indicando el error de manera similar al caso de la inserción.

\begin{figure}

\centering{

\pandocbounded{\includegraphics[keepaspectratio]{../figures/only-access.png}}

}

\caption{\label{fig-operations-only-access}Operaciones de acceso a
elementos.}

\end{figure}%

\subsection{Búsqueda}\label{buxfasqueda}

La Figura~\ref{fig-operations-only-search} contiene la única operación
de búsqueda en una lista enlazada. En el cuadro de texto se inserta el
valor a buscar. Al pulsar el botón se visualizará de manera animada el
recorrido de la lista hasta llegar al valor buscado. Si el valor no se
encuentra en la lista, el cuadro de texto se pondrá de color rojo
indicando que el elemento no fue encontrado. Si el valor es encontrado
entonces el nodo que lo contiene se pondrá de color verde y en el botón
aparecerá la posición.

\begin{figure}

\centering{

\pandocbounded{\includegraphics[keepaspectratio]{../figures/only-search.png}}

}

\caption{\label{fig-operations-only-search}Operaciones de búsqueda.}

\end{figure}%

\subsection{Algoritmos}\label{algoritmos}

En las secciones anteriores se describieron las operaciones básicas
sobre las listas enlazadas. En esta sección se presentan los algoritmos
o construcciones más complejas que actúan sobre las listas. Estos se
basan en las operaciones básicas y proveen un nivel de abstracción
superior.

\begin{figure}

\centering{

\pandocbounded{\includegraphics[keepaspectratio]{../figures/only-algorithms.png}}

}

\caption{\label{fig-operations-only-algorithms}Algoritmos sobre listas.}

\end{figure}%

\begin{enumerate}
\def\labelenumi{\arabic{enumi}.}
\tightlist
\item
  \textbf{Traverse}: recorre cada uno de los elementos de la lista.
  Durante su ejecución, el elemento actual será visualizado de un color
  diferente.
\item
  \textbf{Reverse}: invierte el orden de los elementos de la lista.
\item
  \textbf{Find Middle}: muestra visualmente la ejecución del algoritmo
  que encuentra el elemento que hay en la mitad de la lista.
\item
  \textbf{Get Length}: recorre la lista para contar el número de
  elementos. Este algoritmo tiene el símbolo de advertencia porque,
  aunque es bueno desde un punto de vista pedagógico, no se suele usar
  ya que el número de elementos en la lista se mantiene como parte de su
  estado.
\item
  \textbf{Remove Duplicates}: Muestra de manera animada como se
  encuentran y eliminan los elementos duplicados de una lista.
\end{enumerate}

\section{Herramientas de programador}\label{herramientas-de-programador}

Una característica importante que tiene este módulo de visualización es
la separación de conceptos. Por ejemplo, una lista, conceptualmente, es
una secuencia de elementos. A nivel de programación surgen algunos
conceptos que son importantes a la hora de implementar esta secuencia.
Por ejemplo, se hace necesario hablar de punteros y direcciones de
memoria. Por esta razón el simulador incluye un conjunto de herramientas
en el menú \textbf{Programmer Tools} que se muestra en la
Figura~\ref{fig-programmer-tools}.

\begin{figure}

\centering{

\includegraphics[width=0.5\linewidth,height=\textheight,keepaspectratio]{../figures/programmer-tools.png}

}

\caption{\label{fig-programmer-tools}Vista de herramientas de
programador.}

\end{figure}%

\subsection{Apuntadores al primer y último
nodo}\label{apuntadores-al-primer-y-uxfaltimo-nodo}

El botón \textbf{Show head and tail pointers} hará visibles los punteros
al primer y último elemento en la lista. Hasta ahora las listas que se
han mostrado como la de la Figura~\ref{fig-next-pointer} no tienen esta
opción activada. Es posible apreciar la diferencia con la lista que se
muestra en la Figura~\ref{fig-head-tail}. Los nodos naranja representan
cada uno de los punteros. Si el usuario localiza el puntero del mouse
sobre alguno de ellos la conexión se resaltará.

\begin{figure}

\centering{

\pandocbounded{\includegraphics[keepaspectratio]{../figures/head-and-tail.png}}

}

\caption{\label{fig-head-tail}Visualización de los punteros al inicio y
al final de la lista.}

\end{figure}%

\subsection{Apuntadores adicionales}\label{apuntadores-adicionales}

Durante el diseño de una operación sobre listas es siempre importante
considerar el movimiento de los punteros. Incluso operaciones como
remover un elemento en una posición de la lista usan punteros
adicionales. Para este tipo de operaciones la aplicación cuenta con la
funcionalidad de crear nodos de tipo apuntador. Esto se logra haciendo
click en el botón \textbf{Add pointer} (ver
Figura~\ref{fig-programmer-tools}). Cuando se pulsa este botón se
adiciona un nuevo nodo circular al lienzo y este se puede conectar a
cualquiera de los nodos de la lista.

\begin{figure}

\centering{

\pandocbounded{\includegraphics[keepaspectratio]{../figures/additional-pointers.png}}

}

\caption{\label{fig-additional-pointers}Visualización de los punteros
adicionales insertados por el usuario.}

\end{figure}%

La Figura~\ref{fig-additional-pointers} muestra una lista que incluye
los punteros a la cabeza y a la cola. Adicionalmente también hay dos
punteros llamados \(P\) y \(Q\) que apuntan al segundo y al cuarto
elemento.

Los punteros pueden ser reposicionados utilizando el mouse y arrastrando
la conexión a otro nodo. Haciendo doble click sobre el nodo es posible
cambiar el nombre del puntero. Esto es muy útil en labores de enseñanza.

\section{Otras operaciones}\label{otras-operaciones}

La aplicación cuenta con algunas funcionalidades que no son propias de
las estructuras de datos pero si lo son de la enseñanza y el
aprendizaje. Por ejemplo, en el Tip~\ref{tip-canvas-toolbar} se observa
la barra de herramientas que tiene el lienzo en su parte inferior
izquierda.

\begin{itemize}
\tightlist
\item
  El botón
  \pandocbounded{\includegraphics[keepaspectratio]{../figures/png-export.png}}
  exporta el lienzo en su estado actual a un archivo de imagen con
  extensión \textbf{png}.
\item
  El botón
  \pandocbounded{\includegraphics[keepaspectratio]{../figures/pointer.png}}
  activa la opción de puntero, la cual actúa como un señalador y ayuda a
  la comunicación efectiva de las ideas. El puntero del mouse va a estar
  acompañado de un punto rojo de color atractivo. Para desactivarla se
  debe dar click de nuevo en el mismo botón.
\end{itemize}

\bookmarksetup{startatroot}

\chapter{Trabajo futuro}\label{sec-futuro}

A continuación se presentan algunas ideas que pueden ser desarrolladas
como trabajo futuro en la forma de versiones posteriores.

\begin{enumerate}
\def\labelenumi{\arabic{enumi}.}
\tightlist
\item
  Lasser pointer
\item
  Drawing with tablet on the canvas
\end{enumerate}




\end{document}
